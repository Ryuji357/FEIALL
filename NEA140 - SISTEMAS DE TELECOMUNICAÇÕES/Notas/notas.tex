% !TEX TS-program = pdflatex
% !TEX encoding = UTF-8 Unicode

% This is a simple template for a LaTeX document using the "article" class.
% See "book", "report", "letter" for other types of document.

\documentclass[11pt]{article} % use larger type; default would be 10pt

\usepackage[utf8]{inputenc} % set input encoding (not needed with XeLaTeX)

%%% Examples of Article customizations
% These packages are optional, depending whether you want the features they provide.
% See the LaTeX Companion or other references for full information.

%%% PAGE DIMENSIONS
\usepackage{geometry} % to change the page dimensions
\geometry{a4paper} % or letterpaper (US) or a5paper or....
% \geometry{margin=2in} % for example, change the margins to 2 inches all round
% \geometry{landscape} % set up the page for landscape
%   read geometry.pdf for detailed page layout information

\usepackage{graphicx} % support the \includegraphics command and options

\usepackage[parfill]{parskip} % Activate to begin paragraphs with an empty line rather than an indent

%%% PACKAGES
\usepackage{booktabs} % for much better looking tables
\usepackage{array} % for better arrays (eg matrices) in maths
\usepackage{paralist} % very flexible & customisable lists (eg. enumerate/itemize, etc.)
\usepackage{verbatim} % adds environment for commenting out blocks of text & for better verbatim
\usepackage{subfig} % make it possible to include more than one captioned figure/table in a single float
% These packages are all incorporated in the memoir class to one degree or another...

%%% HEADERS & FOOTERS
\usepackage{fancyhdr} % This should be set AFTER setting up the page geometry
\pagestyle{fancy} % options: empty , plain , fancy
\renewcommand{\headrulewidth}{0pt} % customise the layout...
\lhead{}\chead{}\rhead{}
\lfoot{}\cfoot{\thepage}\rfoot{}

%%% SECTION TITLE APPEARANCE
%\usepackage{sectsty}
%\allsectionsfont{\sffamily\mdseries\upshape} % (See the fntguide.pdf for font help)
% (This matches ConTeXt defaults)

%%% ToC (table of contents) APPEARANCE
\usepackage[nottoc,notlof,notlot]{tocbibind} % Put the bibliography in the ToC
\usepackage[titles,subfigure]{tocloft} % Alter the style of the Table of Contents
\renewcommand{\cftsecfont}{\rmfamily\mdseries\upshape}
\renewcommand{\cftsecpagefont}{\rmfamily\mdseries\upshape} % No bold!

\usepackage{amsmath}
%%% END Article customizations

%%% The "real" document content comes below...

\title{Notas de aula}
\author{Ryuji}
%\date{} % Activate to display a given date or no date (if empty),
         % otherwise the current date is printed 

\begin{document}

\pagenumbering{gobble}
\maketitle
\newpage

\tableofcontents
\newpage

\pagenumbering{arabic}

\section{Radiovisibilidade}

\paragraph{}

\begin{equation}
RSL = P_{T_{X}} - L_{F_{A}} + G_{A} - L_{b} + G_{B} - L_{F_{B}}
\end{equation}
\begin{equation}
RSL -th = FFM
\end{equation}

\subsection{Perda basica de propagação ($L_b$)}

\paragraph{}

\begin{equation}
L_{b} = FSL + A_{a} + D_{L}
\end{equation}

\subsubsection{$FSL$: Free Space Loss}

\paragraph{}

\begin{equation}
FSL = 92,44 + 20log(f \cdot d)
\end{equation}

\textbf{Onde:}

$FSL$ - Perda em [dB] \newline
$f$ - frequencia do sinal em [GHz] \newline
$d$ - distancia entre as antenas em [Km]

\subsubsection{$A_{a}$: Perda de gases e vapores}

\paragraph{}

\begin{equation}
A_{a} = {\gamma}_{a} \cdot d
\end{equation}

\textbf{Onde:}

$\gamma_{a}$ - Da tabela \newline
$d$ - Distancia entre as antenas em [Km]

\subsubsection{$D_{L}$: Perda por obstaculo (sempre simples)}

\paragraph{}

\begin{equation}
D_{L} = 0
\end{equation}

\subsection{Atenuação devido a chuva ($A_{(R, p)}$)}

\paragraph{}

\begin{equation}
A_{(R, p)} = {\gamma}_{(R, p)} \cdot L_{ef}
\end{equation}

\subsection{Qualidade e Disponibilidade}

\paragraph{}

\begin{equation}
U_{sistema} = U_{v}
\end{equation}

\begin{equation}
U_{sistema} = U_{e} + U_{p}
\end{equation}

\subsubsection{$U_{e}$}

\begin{subequations}
\begin{align}
{U}_{e_{(1 + 0)}} & = \frac{MTTR}{MTBF}\\
{U}_{e_{(1 + 1)}} & = \left ( \frac{MTTR}{MTBF} \right ) ^ 2 \\
{U}_{e_{(N + M)}} & = \left ( \frac{N! + M!}{N! \cdot (M + 1)!} \right ) \cdot \left ( \frac{MTTR}{MTBF} \right ) ^{(M + 1)}
\end{align}
\end{subequations}

\textbf{Onde:}

$MTTR$ - Tempo de reparo \newline
$MTBF$ - Tempo entre falhas - Datasheet

\subsubsection{$U_{p}$}

\paragraph{}

\begin{equation}
U_{p} = 0,001\%
\end{equation}

\subsubsection{$U_{v}$}

\paragraph{}

\begin{equation}
U_{v} = 100\% - A_{v}
\end{equation}

\textbf{Onde:}

$A_{v}$ - Da tabela

\end{document}
