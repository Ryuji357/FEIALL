%% abtex2-modelo-trabalho-academico.tex, v-1.9.6 laurocesar
%% Copyright 2012-2016 by abnTeX2 group at http://www.abntex.net.br/ 
%%
%% This work may be distributed and/or modified under the
%% conditions of the LaTeX Project Public License, either version 1.3
%% of this license or (at your option) any later version.
%% The latest version of this license is in
%%   http://www.latex-project.org/lppl.txt
%% and version 1.3 or later is part of all distributions of LaTeX
%% version 2005/12/01 or later.
%%
%% This work has the L PPL maintenance status `maintained'.
%% 
%% The Current Maintainer of this work is the abnTeX2 team, led
%% by Lauro César Araujo. Further information are available on 
%% http://www.abntex.net.br/
%%
%% This work consists of the files abntex2-modelo-trabalho-academico.tex,
%% abntex2-modelo-include-comandos and abntex2-modelo-references.bib

% ------------------------------------------------------------------------
% ------------------------------------------------------------------------
% abnTeX2: Modelo de Trabalho Academico (tese de doutorado, dissertacao de
% mestrado e trabalhos monograficos em geral) em conformidade com 
% ABNT NBR 14724:2011: Informacao e documentacao - Trabalhos academicos -
% Apresentacao
% ------------------------------------------------------------------------
% ------------------------------------------------------------------------
% Personalização para o modelo Udesc 2020 7. ed. revisada e modificada
% MANUAL_2020_09_07_1599489825065_12510.pdf
% Autor: Felipe Joel Zimann (felipezimann@hotmail.com)
% Data: 02/12/2020 v1.0
% Data: 13/02/2021 v1.0.1 alterado tamanho numeração da página para 10pt
% ------------------------------------------------------------------------
% ------------------------------------------------------------------------

\documentclass[
	12pt,				% tamanho da fonte
	openright,			% capítulos começam em pág ímpar (insere página vazia caso preciso)
	oneside,			% para impressão em recto e verso (twoside). Oposto a (oneside)
	a4paper,			% tamanho do papel. 
	chapter=TITLE,		% títulos de capítulos convertidos em letras maiúsculas
	section=TITLE,		% títulos de seções convertidos em letras maiúsculas
	sumario=abnt-6027-2012,
	english,			% idioma adicional para hifenização
	brazil,				% o último idioma é o principal do documento
	fleqn,				% equações alinhadas a esquerda (UDESC/CCT)+
	]{abntex2}

\input{PacotesBasicos}	% Incliu pacotes básicos 

% -----------------------------------------------------------------
% Você pode adicionar seus pacotes a partir desta linha;
% -----------------------------------------------------------------

%\usepackage[showframe,pass]{geometry}
%\usepackage[11,12]{pagesel}

% -----------------------------------------------------------------
% Informações de dados para CAPA e FOLHA DE ROSTO
% -----------------------------------------------------------------

\titulo{Título do Trabalho}%

\autor{Nome do Autor {}Sobrenome}%
\orientador{Nome do orientador{} Sobrenome}%
\coorientador{Nome do coorientador{} Sobrenome}%

% ATENÇÃO: O símbolo {} indica o sobrenome para a ficha catalográfica.
% Exemplo: Sherlock Holmes {}da Silva para sobrenomes compostos;
% Exemplo: Arnold Alois {}Schwarzenegger para sobrenome simples.

\instituicao{Universidade do Estado de Santa Catarina, Centro de Ciências Tecnológicas, Programa de Pós--Graduação em Engenharia Elétrica}%

%\tipotrabalho{Tese (Doutorado)}
\tipotrabalho{Dissertação (Mestrado)}

%\preambulo{Tese apresentada ao Programa de Pós--Graduação em Engenharia Elétrica do Centro de Ciências Tecnológicas da Universidade do Estado de Santa Catarina, como requisito parcial para a obtenção do grau de Doutor em Engenharia Elétrica.}

\preambulo{Dissertação apresentada ao Programa de Pós--Graduação em Engenharia Elétrica do Centro de Ciências Tecnológicas da Universidade do Estado de Santa Catarina, como requisito parcial para a obtenção do grau de Mestre em Engenharia Elétrica.}

\local{Joinville}%

\data{\the\year}%
% ---

% compila o indice
\makeindex

% -----------------------------------------------------------------
% Início do documento
% -----------------------------------------------------------------
\begin{document}

\selectlanguage{brazil}
\frenchspacing  % Retira espaço extra obsoleto entre as frases.

% -----------------------------------------------------------------
% ELEMENTOS PRÉ-TEXTUAIS
% -----------------------------------------------------------------
\pretextual

% Você pode comentar os elementos que não deseja em seu trabalho;

% A capa pode ser escolhida dentro do arquivo Capa.tex (TCC, Master, Doc, ...)
\include{PreTextuais/Capa}					% Elemento Obrigatório
\include{PreTextuais/FolhadeRosto}			% Elemento Obrigatório
% Caso não utilize a Ficha Catalográfica entre na folha de rosto e retire o * de dentro do arquivo FolhadeRosto
\include{PreTextuais/FichaCatalografica}	% Elemento Obrigatório (Verso da Folha)
\include{PreTextuais/Errata}				% Elemento Opcional
\include{PreTextuais/FolhadeAprovacao}		% Elemento Obrigatório
\include{PreTextuais/Dedicatoria}			% Elemento Opcional
\include{PreTextuais/Agradecimentos}		% Elemento Opcional
\include{PreTextuais/Epigrafe}				% Elemento Opcional
\include{PreTextuais/Resumo}				% Elemento Obrigatório
\include{PreTextuais/Abstract}				% Elemento Obrigatório
\include{PreTextuais/Listas}				% Elemento Opcional
\include{PreTextuais/Sumario}				% Elemento Obrigatório

% -----------------------------------------------------------------
% ELEMENTOS TEXTUAIS
% -----------------------------------------------------------------
\textual

\pagestyle{PagNumReduzida}						% Comando para cabeçalho somente com numeração de página 10pt
\aliaspagestyle{chapter}{PagNumReduzida}		% Deixar numeração da primeira página com tamanho igual ao resto da numeração
% ref.: https://groups.google.com/g/abntex2/c/CP7g8ZMgi-c/m/KjfEnn5b9a4J


% ---- Mantenha está estrutura, assim você deixa o trabalho mais organizado -------

\include{Textuais/Capitulo01}
%\include{Textuais/Capitulo02}
%\include{Textuais/Capitulo03}

% -----------------------------------------------------------------
% ELEMENTOS PÓS-TEXTUAIS
% -----------------------------------------------------------------
\postextual

% Você pode comentar os elementos que não deseja em seu trabalho;

% Referências bibliográficas
\bibliography{abntex2-ref_UDESC_2020}	% Elemento Obrigatório

\include{PosTextuais/Glossario}				% Elemento Opcional
\include{PosTextuais/Apendices}				% Elemento Opcional
\include{PosTextuais/Anexos}				% Elemento Opcional
\include{PosTextuais/IndiceRemissivo}		% Elemento Opcional

\end{document}

% -----------------------------------------------------------------
% Fim do Documento
% --------------------------------------------------------