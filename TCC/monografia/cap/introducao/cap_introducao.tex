\chapter{Introdução}

\section{Objetivo}

Compreender e aplicar o conceito da tecnologia LPT (\emph{Laser Power Transmission}) a fim de desenvolver um dispositivo móvel capaz de transmitir energia sem fio, que seja compacto e seguro, para atender as mais diversas aplicações de baixo consumo energético, baseando-se em estudos já existentes.

\section{Justificativa}

Portabilidade tem sido uma das grandes evoluções da tecnologia nos últimos 30 anos principalmente em relação aos celulares, notebooks e mais recentemente a tablets. Seguindo a evolução dos telefones fixos para os celulares e dos computadores desktop para os notebooks de apenas 1kg, nota-se que mobilidade se tornou uma característica chave de nossas tecnologias [1]. Quando os primeiros celulares chegaram ao Brasil em 1990 eram apenas 667 de aparelhos, já em 2010 passaram a ser 197,53 milhões [2] e em 2020 passaram a ser 230 milhões [3]. Os novos celulares, tablets e notebooks têm sido lançados com baterias mais duráveis e com circuitos cada vez mais energeticamente eficientes, porém ainda é necessário o carregamento da bateria destes dispositivos uma vez por dia. Outro ponto a ser considerado é a degradação notória das baterias por conta do uso contínuo [4]. Nota-se também a grande tendência do mundo wireless (sem fio), a grande maioria das marcas já estão oferecendo fones bluetooth e celulares com carregadores indutivos, porém estes necessitam estar perto de uma tomada. O próximo passo da portabilidade desses aparelhos está relacionado completamente à forma como vamos carregar suas baterias, com isso a tecnologia LPT mostra-se como uma possível solução de wireless charging para esta questão. No mercado já existem produtos de wireless charging, um deles sendo da empresa israelita Wi-Charge, criada em 2012, onde seus produtos são focados para aplicações de baixa potência como celulares, torneiras elétricas automáticas e aparelhos de smart home [5].