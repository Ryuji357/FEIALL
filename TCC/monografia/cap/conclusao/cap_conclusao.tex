\chapter{Conclusão}

Inicialmente o projeto foi idealizado com o objetivo de transferir energia utilizando um laser comercial. Entretanto, a necessidade de desenvolver um sistema de segurança impossibilitou a utilização de um laser convencional, o que aumentou consideravelmente a complexidade do trabalho sendo necessário o desenvolvimento de uma cavidade de ressonância externa para que o projeto pudesse ser concebido.

A dificuldade em realizar uma simulação relaciona-se com a necessidade do ganho médio, outro fator de grande dificuldade e que afetou o desenvolvimento da simulação do projeto foi a limitação dos softwares disponíveis no mercado. Além disso, essa nova forma de transmissão é muito recente e não há bibliografias especificas sobre o assunto disponíveis, apenas poucos artigos científicos.

Os resultados obtidos demonstram que o dispositivo apresenta uma baixa eficiência em decorrência de uma alta densidade de potência focada na célula da placa que afetou negativamente o desempenho da célula, entretanto pode-se confirmar que a perda em relação a distância é praticamente nula, o que mostra que raios lasers podem ser utilizados para transferir energia em grandes distâncias.

Apesar dos dados abaixo do esperado, o projeto continua sendo promissor, pois a ideia de transmitir energia de forma sem fio ainda é muito atrativa, principalmente em ambientes de smart home, lojas e shoppings. Além de haver espaço para o desenvolvimento de novas e melhores placas fotovoltaicas, aumentando a eficiência do conjunto.